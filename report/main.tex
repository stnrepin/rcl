\documentclass[a4paper,14pt]{extarticle}
\usepackage{rotating}
\usepackage{verbatim}
\input{inc/preamble.tex}

\usepackage[outputdir=out,cache=false]{minted}

\begin{document}

\input{inc/titlepage}

\renewcommand*{\thepage}{}
\tableofcontents
\clearpage
\renewcommand*{\thepage}{\arabic{page}}

\setcounter{page}{3}

\section{Задание}

Реализовать алгоритм Remote Core Blocking. Разработать простой тест. Оценить
эффективность.

\section{Выполнение работы}

Алгоритм Remote Core Locking (RCL) написан на языке C11 для ОС Linux x64 на
основе публикации \cite{lozi12}.

RCL -- это алгоритм блокировки, обеспечивающий высокую производительность на
многоядерных системах за счет исполнения критических секций на
специализированных ядрах, заменяя захват блокировки на вызов процедуры. Такой
подход повышает эффективность использования L1 и L2 кэшей процессора.

Кроме того был реализован тест, запускающий $N$ потоков, каждый из которых
инкрементирует общую переменную $M$ раз на величину $S$. По завершению тестовая
программа сверяет полученное значение переменной с ожидаемым ($N\cdot M\cdot
S$). Измеряется время с момента запуска потоков до их завершения.

На графике ниже показана зависимость времени выполнения от числа потоков для
двух реализаций тестовой программы: с использованием RCL и мьютексов. Из
графика следует, что с ростом contention, когда все больше потоков соревнуются
за доступ к критической секции, RCL начинает выигрывать реализацию на
мьютексах. При малом числе потоков мьютексы оказываются быстрее из-за накладных
расходов RCL.

Параметры стенда представлены в таблице ниже:

\begin{table}[H]
\centering
\begin{tabular}{|c|c|}
    \hline
    Название & Значение \\
    \hline
    ОС & Arch Linux (Linux 5.16 x64) \\
    \hline
    Процессор & Intel Core i7 11700 \\
    \hline
    Число ядер & 16 (8) \\
    \hline
\end{tabular}
\end{table}

\addimghere{res/plot.png}{1}{}{}

\section{Разделение работы в бригаде}

Вместе выяснили и обсудили суть алгоритма, особенности его реализации.
Определились с интерфейсом библиотеки и ее тестированием.

Дмитрий Николаев выполнил разработку тестовой утилиты, скриптов для анализа
результатов и Servicing Thread.

Степан Репин реализовал остальные части библиотеки: Manager Thread, Backup
Thread, инициализацию, запуск. А также выполнил отладку.

\begingroup
\renewcommand{\section}[2]{\anonsection{Список использованных источников}}
\begin{thebibliography}{00}

\bibitem{lozi12}
    J.-P. Lozi, F. David, G. Thomas, J. Lawall, G. Muller. Remote Core Locking:
    Migrating Critical-Section Execution to Improve the Performance of
    Multithreaded Applications // USENIX ATC 12. - 2012. - P. 65-76

\end{thebibliography}
\endgroup

\end{document}

